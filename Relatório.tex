% Setup -------------------------------

\documentclass[a4paper]{report}
\usepackage[a4paper, total={6in, 10in}]{geometry}
\setcounter{secnumdepth}{3}
\setcounter{tocdepth}{3}

\usepackage{hyperref}
\usepackage{tabularx}
\usepackage{booktabs}

% Encoding
%--------------------------------------
\usepackage[T1]{fontenc}
\usepackage[utf8]{inputenc}
%--------------------------------------

% Portuguese-specific commands
%--------------------------------------
\usepackage[portuguese]{babel}
%--------------------------------------

% Hyphenation rules and references
%--------------------------------------
\usepackage{hyphenat}
\usepackage{hyperref}
%--------------------------------------

% Capa do relatório

\title{
   Desenvolvimento de Aplicações WEB
    \\ \Large{\textbf{Trabalho Prático}}
    \\ -
    \\ Mestrado em Engenharia Informática
    \\ \large{Universidade do Minho}
    \\ Relatório
}
\author{
    \begin{tabular}{ll}
        \textbf{Grupo nº}
        \\\hline
        PG & Miguel
        \\
        PG41094 & Pedro Rafael Paiva Moura
        \\
        A80499  & Moisés Manuel Borba Roriz Ramires
    \end{tabular}
}

\date{\today}

\begin{document}

\begin{titlepage}
    \maketitle
\end{titlepage}

% Resumo

\begin{abstract}
    Neste trabalho prático foi nos pedido que desenvolvêssemos uma aplicação web que iria servir como  rede social ,de alunos para alunos, onde estes pudessem partilhar materiais, discutir datas, combinar eventos...
\end{abstract}

% Índice

\tableofcontents

% Introdução

\chapter{Introdução} \label{intro}
\large{
    Neste projeto foi-nos proposto a realização de uma rede social com algumas funcionalidades integradas,em seguida destacadas:
    \begin{enumerate}
        \item Gestão de utilizadores com autenticação. Para aceder á página é necessário que um utilizador esteja autencticado;
        \item Publicações por parte dos utilizadores. Um utilizador deve poder publicar na página desde que estaja autenticado;
        
    \end{enumerate}
    Para além desta funcionalidade foram acrescentadas mais algumas á aplicação.
}



\end{document}