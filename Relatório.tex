% Setup -------------------------------

\documentclass[a4paper]{report}
\usepackage[a4paper, total={6in, 10in}]{geometry}
\setcounter{secnumdepth}{3}
\setcounter{tocdepth}{3}

\usepackage{hyperref}
\usepackage{tabularx}
\usepackage{booktabs}

% Encoding
%--------------------------------------
\usepackage[T1]{fontenc}
\usepackage[utf8]{inputenc}
%--------------------------------------

% Portuguese-specific commands
%--------------------------------------
\usepackage[portuguese]{babel}
%--------------------------------------

% Hyphenation rules and references
%--------------------------------------
\usepackage{hyphenat}
\usepackage{hyperref}
%--------------------------------------

% Capa do relatório

\title{
   Desenvolvimento de Aplicações WEB
    \\ \Large{\textbf{Trabalho Prático}}
    \\ -
    \\ Mestrado em Engenharia Informática
    \\ \large{Universidade do Minho}
    \\ Relatório
}
\author{
    \begin{tabular}{ll}
        \textbf{Grupo nº}
        \\\hline
        PG & Miguel
        \\
        PG41094 & Pedro Rafael Paiva Moura
        \\
        A80499  & Moisés Manuel Borba Roriz Ramires
    \end{tabular}
}

\date{\today}

\begin{document}

\begin{titlepage}
    \maketitle
\end{titlepage}

% Resumo

\begin{abstract}
    No âmbito da cadeira Desenvolvimento de aplicações WEB, do ano letivo 2019/2020, foi-nos proposto como trabalho prático que desenvolvêssemos uma aplicação web que iria servir como  rede social de alunos para alunos, onde estes pudessem partilhar materiais, discutir datas, combinar eventos... . Para tal irão ser utilizadas as ferramentas e técnicas utilizadas/desenvolvidas durante o decorrer do semestre.
\end{abstract}

% Índice

\tableofcontents

% Introdução

\chapter{Introdução} \label{intro}
\large{
    Neste projeto foi-nos proposto a realização de uma rede social com algumas funcionalidades integradas,em seguida destacadas:
    \begin{enumerate}
        \item Gestão de utilizadores com autenticação. Para aceder á página é necessário que um utilizador esteja autencticado;
        \item Publicações por parte dos utilizadores. Um utilizador deve poder publicar na página desde que estaja autenticado;
        
    \end{enumerate}
    Para além desta funcionalidade foram acrescentadas mais algumas à aplicação, como comentários sobre publicações, uma pagina do perfil do aluno,um aluno poder seguir tags, uma publicação poder ter ficheior anexados um chat geral, entre outras.
}

\chapter{Autenticação e acesso à aplicação}
    Tal como referido anteriormente, é necessário utilizar um método para que os alunos possam aceder à aplicação, e confirmar que um aluno não registado( ou que não tenha feito log in) não tenha acesso as diversas funcionalidades do programa.
    Para tal foram utilizado métodos como passport, sessions e jwtokens, e para verificação de que "estrangeiros" não tenham acesso á app foi criada uma função que Verifica a autnticação, caso o aluno não se encontre autenticado  é remetido para a paginal inicial.

\end{document}